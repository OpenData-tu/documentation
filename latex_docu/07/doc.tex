\section{Database}\label{database}

\textbf{Authoriship}

\begin{longtable}[]{@{}llll@{}}
\toprule
Version & Date & Modified by & Summary of changes\tabularnewline
\midrule
\endhead
0.2 & 2017-07-28 & Tasche, Nico & changed to new
structure\tabularnewline
0.1 & 2017-07-24 & Tasche, Nico & first working draft\tabularnewline
\bottomrule
\end{longtable}

\subsection{Overview}\label{overview}

\subsection{Requirements}\label{requirements}

\subsubsection{Requirements}\label{requirements-1}

\begin{itemize}
\tightlist
\item
  scale-able in the range of petabyte in size
\item
  hundred of thousands of requests per minute
\item
  high availablility
\item
  partition tolerance
\item
  fast to handle timeseries
\item
  fast to handle geolocation data
\item
  immediate consistency is NOT necessary
\end{itemize}

As seen in the requirements, a hugh focus was on scalability. We had
some secondary reqirements as well, which were mainly regarding our
possibilities to handle the project. \#\#\# Requirements - Open source
or at the very least an open license is required. - must be well
documented - must be managable regarding administration and learning
afford

\subsection{Survey of Existing
Solutions}\label{survey-of-existing-solutions}

\subsection{Evaluation Criteria \& Decision-making
Process}\label{evaluation-criteria-decision-making-process}

The process of deciding what database architecture to use we started
with our requirements.

Espacialy the last point of our secondary requiremnts had to be taken
into account, because we had no real database expert in our team, so we
first considert database-system we allready knew. Our approach was to
check whether those databases fulfill our requirements first.

Any relational database as main datastorage has been quickly
disreagarded, cause of the bad fairly scaling behavir with the amount of
data we have to handly. \#\# Implementation Details \#\#\# Intro to
Elasticsearch Elasticsearch is an opensource Lucene based search engine.
It is under active development, with an extensive documentation. \#\#\#
Architecture Each index can be sharded and each shard can have multiple
indieces.

TODO: Picture of architecture

To better distribute search requests, the workload is divided among all
shards belonging to an index. Because that would not scale very well and
would have no partition tolerance, each shard has a configurable number
of replicas. A new search request is send to on replica of each shard.

\subsubsection{Data model}\label{data-model}

We decided to have an data model which is data-source-centric with the
extra posibility to partition the data over time. That means, each data
source gets it own index with its own timeframe and its own adjusted
datastructure. All our data sources save a few basic data point with
each element stored in the database, in particular are those: -
timestamp: when has the datapoint been recorded - location: where has
the datapoint been recorded

Those are acutally the only information we need to store, besides the
individual measurements. We do acutally store some more information, but
regarding the common usecases for searches those two datapoints are
enough for environemental data. Please refer to the full data-model in
the appedix for more information.

This data-model has multiple advantages: - it keeps the data provenance
- it allows us to adjust the server infrastructe based on the data
source - it scales indefinitely - index size is deterministic, cause of
time based partitioning

So why does it scale so good? When importing data from one source, I
process and store the data points in one index. This index is not just
limited to the data source, it is also limited to the time, e.g.~2016.
That means, when 2016 is finished with importing data, the index is done
and can be closed up, no one needs to care about it anymore. After the
index is done, it might even be transfered to another Elasticsearch node
with different hardware. That would be usefull, for example, when the
average density of the smurf population is beeing stored. The index can
be transferd to a less powerfull hardware with fewer CPU cores and
spinning harddrives and even fewer replicas, because this information is
probably hardly requested.

\subsubsection{Query optimization}\label{query-optimization}

Why do we need query optimization? For that I'm going to give a small
small example to consider: 1. we import multiple sources, with multiple
messurements: source1(airtemperature, watertemperatur) 1980-2017,
source2(airtemperature) 1983-1990, source3(uv-index) 2009-2017 2. each
source is partitioned by year and source 1 is partitioned by month for
all data after 2015. 3. every index is naivly sharded over 3 nodes

Let's make a simple search request: give me all uv values data from 2015
till 2017 and aggregate an everage over the month. Because the user does
not now anything about the internal database architecture (at least he
should not) he requests the temperature and the timeframe.

\paragraph{Worst case:}\label{worst-case}

A search request in send to all indieces, that means:

\begin{verbatim}
source1 = 35 years + (2years x 12 month) x 3 shards
source1 = 177 shards

source2 = 17 years x 3 shars
source2 = 51 shards

source3 = 8 years x 3 shars
source3 = 8 years x 3 shars

source1 + source2 + source3 = 252 shards
\end{verbatim}

So in worst case each shard has its own node(very unlikely), the search
request has to be send to 252 nodes/computers.

\paragraph{First optimization, Limit the
time}\label{first-optimization-limit-the-time}

With a naive approach by checking the common time part of the request
2016-2017 and limit the indieces search with the following pattern:

\begin{verbatim}
indexsearch: *-201*
\end{verbatim}

\begin{verbatim}
source1 = 5 years + (2years x 12 month) x 3 shards
source1 = 87 shards

source2 = 0 years x 3 shars
source2 = 0 shards

source3 = 3 years x 3 shars
source3 = 9 shards

source1 + source2 + source3 = 96 shards
\end{verbatim}

We allready reduced the number of shards we need to address by 61\%

With a little more sophisticated timelimitation algorithm, we could
acctually do more and search just those two years:

\begin{verbatim}
indexsearch: *-2016, *-2017
\end{verbatim}

\begin{verbatim}
source1 = 1 years + (2years x 12 month) x 3 shards
source1 = 75 shards

source2 = 0 shards

source3 = 2 years x 3 shars
source3 = 6 shards

source1 + source2 + source3 = 81 shards
\end{verbatim}

Now we are at 68\% reduction.

\paragraph{Second optimization, Limit to indieces which contain the
right
data}\label{second-optimization-limit-to-indieces-which-contain-the-right-data}

If we store in a seperate database, which data source and therefore
indiece actually holds the requested data we can do even much more:

\begin{verbatim}
indexsearch: source3-2016, source3-2017
\end{verbatim}

\begin{verbatim}
source1 = 0
source2 = 0 shards
source3 = 2 years x 3 shars = 6 shards

source1 + source2 + source3 = 6 shards
\end{verbatim}

By using those two optimitzations, we were able to reduce the number of
requeseted shard to 6, which means a total reduction of 96.2\%.

This was just a naive example. In reality the reduction should even be
much higher, with a growing number of data sources. Let's say we have
allready 100 data sources and we can limit a request to just two of
those for example because the requested messuremnt it provided just by
those two, the saving of network traffic and workload would be immense.

\subsection{Critical
Analysis/Limitations}\label{critical-analysislimitations}

\subsubsection{Joins}\label{joins}

One mayor drawback of elasticsearch is the missing possibility of server
side join, the way they are known by SQL based database-system. This
means, any kind of join operation has to be done either on a seperate
server, like our api instance, or on the application side. This is
actually something we were not really aware of for a long time.

\subsection{Future Development and
Enhancements}\label{future-development-and-enhancements}
