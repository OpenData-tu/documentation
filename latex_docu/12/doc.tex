\section{Conclusions}\label{conclusions}

\textbf{Authoriship}

\begin{longtable}[]{@{}llll@{}}
\toprule
Version & Date & Modified by & Summary of changes\tabularnewline
\midrule
\endhead
0.1 & 2017-07-19 & Oliver, Andres & Working draft\tabularnewline
\bottomrule
\end{longtable}

\subsection{Is Open Data really open?}\label{is-open-data-really-open}

\begin{itemize}
\tightlist
\item
  What sources did we ``unlock''
\item
  Data source types (ftp, http, REST)
\item
  Different data formats (CSV, XLS, HTML)
\item
  Different measurands (Temperature, air pollution, etc)
\end{itemize}

//\texttt{TODO} look into https://www.wired.com/1994/11/agre-if-2/ as a
possible reference

\subsection{Lessons learned}\label{lessons-learned}

\begin{itemize}
\tightlist
\item
  Maslow's hammer: ``when you have a hammer, everything looks like a
  nail''
\item
  Database choice: Start with use cases \& (types of) queries
\item
  Test your ideas on real infrastructure rather conjecturing
\item
  Balance using existing knowledge vs.~investing time in learning new
  technologies
\item
  Never underestimate overhead of cloud resource procurement,
  configuration \& deployment
\item
  Availability of ``open'' data is virtually inexistent for automated
  agents
\item
  Most provide a GUI, but not the data
\end{itemize}

\subsubsection{Team \& Process}\label{team-process}

\begin{itemize}
\tightlist
\item
  Because English was not the native language of the majority of the
  team members, this led to difficulties with preciseness when having
  discussions and reaching consensus/conclusions
\item
  Cultural differences (indirect speech, etc.) also contributed to
  protracted discussions and difficulties reaching agreement, in
  general.
\item
  Team members tended to work separately. This in itself is not
  generally a problem, but in our case, due to lack of communication \&
  synchronization with the rest of the team, led to end-products which
  sometimes did not work together well, or required significant efforts
  to integrate and harmonize. bugs or issues in the individual
  components were not discovered until quite late in the process (until
  deployment). Further, because the infrastructure deployment rested on
  the shoulders of a single member, this created a bottleneck and
  excessive load on this person.
\item
  Development tools were established from the get-go, but somewhere
  around the midpoint of the project timeline most stopped using Trello
  to keep track of tasks and TODOs. Obviously this led to lack of
  visibility as to what was to be done, and contributed to the general
  displacement of team members into silos.
\end{itemize}

\subsection{Future improvements}\label{future-improvements}

-- What are the problems with the architecture as it is today -- What
was \emph{not} implemented and WHY

\begin{itemize}
\tightlist
\item
  Postprocessing of indices/data into measurand-based hierarchy
\item
  Optimize queries with metadata from relational database
\item
  Ability to recover importer from failure via checkpointing
\item
  User-defined time schedule for importers
\item
  Auto-scaling of importers based on resource requirements
\end{itemize}
