\section{Infrastructure}\label{infrastructure}

\textbf{Authorship} Written by Andres

\subsection{Overview/Description (if applicable
Motivation)}\label{overviewdescription-if-applicable-motivation}

Needless to say, infrastructure plays a critical role in any cloud-based
system. Despite the critical nature of the availability of
infrastructure to a cloud prototyping

\subsection{Requirements}\label{requirements}

Based on the particular factors discussed, we identified the following
points as being important requirements as part of our infrastructure:

\begin{enumerate}
\def\labelenumi{\arabic{enumi}.}
\tightlist
\item
  \textbf{Platform agnostic components}: Our infrastructure must be
  deployable to both public and private clouds. Therefore, making use of
  provider-specific features was not possible.
\item
  \textbf{Budget}: The choice of cloud provider should allow us to
  prototype and test in a shoestring budget (US\$100 for the whole
  semester).
\end{enumerate}

In addition, the general requirements for the system with regards to
scalability, performance, etc. also apply to this component, as
mentioned.

\subsection{Survey of Existing Solutions (available
implementations)}\label{survey-of-existing-solutions-available-implementations}

\subsection{Evaluation Criteria \& Decision-making
Process}\label{evaluation-criteria-decision-making-process}

\subsubsection{Cloud Platform}\label{cloud-platform}

The cloud-agnostic nature of our system would have allowed us to deploy
prototypes to any cloud provider to which we had access. Due to the
unavailability of cloud resources to our project, however, we had a
single choice when it came to deciding on the platform on which to
deploy our infrastructure, namely AWS since it was the only platform for
which we had credits. Conducting a thorough comparison of cloud
platforms which we subsequently wouldn't have been able to use, did not
seem like a good use of our time.

\subsubsection{Orchestration}\label{orchestration}

\subsection{Implementation Details}\label{implementation-details}

\subsubsection{Orchestration}\label{orchestration-1}

Given its prominence in the cloud arena and the fact that it is open
source software, we used Kubernetes for orchestration of cloud
components. Having containerized data importers was a requirement from
early on in order to accomplish scalability, and Kubernetes supports the
creation and monitoring of containers natively. In addition, the
building blocks of our architecture such as the queue and the database,
which must be guaranteed to be up and running

\subsubsection{Deployment Automation}\label{deployment-automation}

Due to the very specific constrains that our project found itself in, a
disproportionately large portion of the effort went into managing
infrastructure. Having no budget meant that whatever infrastructure was
deployed had to be immediately torn down after it was no longer needed.

Difficulties notwithstanding, this meant that the automation process for
deployemnt got well refined, with robust scripts, and the addition of
Kubernetes Operations (kops) and its native support for AWS translated
into an even smoother deployment experience.

Spot instances also played an important role in keeping to the budget,
and support for these were was integrated into the scripts, which
allowed us to monitor and specify the price for spot instances.

\subsection{Evolution of Component during development (Reasons for the
changes)}\label{evolution-of-component-during-development-reasons-for-the-changes}

\subsection{Discussion/Analysis/Limitations}\label{discussionanalysislimitations}

\subsection{Future
Development/Enhancements}\label{future-developmentenhancements}
