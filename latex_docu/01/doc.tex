\section{Introduction}\label{sec:introduction}

\textbf{Authorship}

\begin{longtable}{@{}llll@{}}
\toprule
Version & Date & Modified by & Summary of changes \\
\midrule
\endhead
0.1.0 & 2017-07-19 & Oliver, Andres & Outline \& working draft \\
0.1.1 & 2017-07-20 & Paul & Motivation section first draft \\
0.2.0 & 2017-07-22 & Oliver, Andres & Motivation rewrite \& flesh out \\
			&						 &								& Requirements section \\
0.2.1 & 2017-07-29 & Oliver, Andres & Fixes for Markdown -- LaTeX conversion \\
0.2.2 & 2017-07-29 & Oliver & Moved Organization content\\
0.3.0 & 2017-07-30 & Oliver, Andres & Cleanup before hand-in \\
\bottomrule
\end{longtable}

\subsection{What is Open Data}\label{what-is-open-data}

So what do we mean when we talk about \emph{open data}?

\begin{quotation}
``Open means anyone can freely access, use, modify, and share for any
purpose (subject, at most, to requirements that preserve provenance and
openness).'' http://opendefinition.org/
\end{quotation}

Further, the definition encompasses the following aspects:

\begin{quote}
\begin{itemize}
\tightlist
\item
  \textbf{Availability and Access}: the data must be available as a
  whole and at no more than a reasonable reproduction cost, preferably
  by downloading over the Internet. The data must also be available in a
  convenient and modifiable form.
\item
  \textbf{Re-use and Redistribution}: the data must be provided under
  terms that permit re-use and redistribution including the intermixing
  with other datasets.
\item
  \textbf{Universal Participation}: everyone must be able to use, re-use
  and redistribute - there should be no discrimination against fields of
  endeavour or against persons or groups. For example, `non-commercial'
  restrictions that would prevent `commercial' use, or restrictions of
  use for certain purposes (e.g.~only in education), are not allowed.\\
  http://opendatahandbook.org/guide/en/what-is-open-data/
\end{itemize}
\end{quote}

So how does this ``open data'' differ from the data we're interested in?
The simple answer is the lack of the time dimension in existing data.
The large majority of open data out there today is static. Environmental
data are usually represented on discrete ``datasets'', but there is no
connection between data of the same source or type which was collected
at different times. For example, one might find a dataset about a
particular environmental measurement such as average and maximum water
pollution on a given set of geographical areas or points. However, the
same data for the following month or year are published separately, with
no connection to the first, and at times having different formats or
semantics.

We therefore distinguish between static data of the type described above
(of which there is an abundance) to the data which are in scope for the
project, namely time series data of environmental measurements coming
from a device (i.e.~a sensor or sensor network) with some form of
geo-information. For example, an array of air pollution sensors in a
given city may collect data at 15-minute intervals; this would be
represented as individual records containing the timestamp, the sensor's
geolocation and the air pollution measurement.

\subsection{Motivation}\label{motivation}

`Open data' efforts in the past have focused on providing a
centralized platform onto which data producers can upload their data
along with some metadata but without much (if any) concern for the
schema or the format in which the data is offered. This has resulted in
very large independent and heterogeneous catalogs of data which are
difficult to discover and integrate without significant manual effort.

The advent of the Internet of Things (IoT) has also meant that
unprecedented amounts of data are generated by millions of devices every
second of every day. However, devices generate data in their own (often
closed-source proprietary) format, thanks to a lack of a common or
widely established data model for environmental data. This results in
lots of data from which it's difficult to gain insights due to the
inherent difficulty in querying data in disparate representations.

Also, as the price of devices declines, more and more enthusiasts are
willing to share their data to open communities so that it can be used
and queried by anyone. To tackle these challenges, our project centers
around building a prototype which provides:

\begin{itemize}
\tightlist
\item
  a platform on which data owners can share their sensor-generated
  environmental data,
\item
  a unified schema to support queries across different data from
  heterogeneous sources
\item
  a simple and extensible framework to facilitate the data import
  process,
\item
  and a flexible querying interface for accessing the data.
\end{itemize}

We therefore aim to create a tool that generates economic and social
value through new and creative ``layering'' of data.

\subsubsection{Building Blocks}\label{building-blocks}

The problem domain can be decomposed into the following architectural
building blocks:

\begin{enumerate}
\def\labelenumi{\arabic{enumi}.}
\tightlist
\item
  Data Import Framework
\item
  Database
\item
  Public API
\item
  Cloud Infrastructure
\end{enumerate}

\subsection{Requirements}\label{requirements}

The above objectives translate into the following requirements:

\subsubsection{General}\label{general}

\begin{enumerate}
\def\labelenumi{\arabic{enumi}.}
\tightlist
\item
  \textbf{Open Source Software}: Libraries \& components used shall be
  open source software.
\item
  \textbf{Cloud architectural style}: Guiding architectural principle
  shall be to avoid monolith-style applications, but rather include
  cloud concerns from early on (i.e.~design phase).
\item
  \textbf{Scalability}: Components shall be inherently scalable.
\item
  \textbf{Fault tolerance}: Components shall provide fault tolerance
  capabilities.
\item
  \textbf{Performance}: The system (and its constituent components)
  shall have the ability to handle extremely high demand.
\item
  \textbf{Portability}: The system shall be deployable both on public
  and private clouds.
\end{enumerate}

\subsubsection{Data Import Framework}\label{data-import-framework}

\begin{enumerate}
\def\labelenumi{\arabic{enumi}.}
\tightlist
\item
  Provide facilities for common access patterns of data sources
  (e.g.~FTP, HTTP)
\item
  Provide facilities to read data in common formats (e.g.~JSON, CSV,
  XML)
\item
  Provide facilities to map the user schema to the platforms common
  schema
\item
  Provide reusable community-based unit converters
\end{enumerate}

\subsubsection{Database}\label{database}

\begin{enumerate}
\def\labelenumi{\arabic{enumi}.}
\tightlist
\item
  Ability to perform time series and geolocation range queries
\end{enumerate}

\subsubsection{Public API}\label{public-api}

\begin{enumerate}
\def\labelenumi{\arabic{enumi}.}
\tightlist
\item
  RESTful

  \begin{itemize}
  \tightlist
  \item
    Stateless
  \end{itemize}
\end{enumerate}
