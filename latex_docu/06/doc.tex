\section{Inserting into Elasticsearch}\label{sec:inserting}

\textbf{Authorship}

\begin{longtable}[]{@{}llll@{}}
\toprule
Version & Date & Modified by & Summary of changes\tabularnewline
\midrule
\endhead
0.1 & 2017-07-24 & Paul & initial version\tabularnewline
0.2 & 2017-07-30 & Andres & Proofread\tabularnewline
\bottomrule
\end{longtable}

This section covers our consumer component, which is responsible for
taking data items from the queue and inserting them into the main
database. We will discuss the requirements and the tasks it fulfills.

\subsection{Requirements}\label{requirements}

In our architectural design, data importers push their output to a
queue, in our case \emph{Apache Kafka}. From the queue, the data items
must be inserted into Elasticsearch. Besides simply taking data from the
queue there are additional validation requirements that this component
fulfills.

\subsection{Validation}\label{validation}

The output generated by data importers (JSON documents) that are
prepared and processed must be validated prior to insertion in the
database. That task is logically linked to the data storage component
and not the importing process itself since:

\begin{itemize}
\tightlist
\item
  the ETL framework should not validate its own outcome (as it or its
  creator are biased)
\item
  the database should ensure it is not importing corrupted datapoints
  that will break queries, etc.
\end{itemize}

The logical place to do validation was then the component that first
processes the data items after the importers have written them to the
queue.

\subsubsection{Ability to scale and adjust the configuration according
to Kafka's
configuration}\label{ability-to-scale-and-adjust-the-configuration-according-to-kafkas-configuration}

Kafka can be configured in many ways. This mainly is about distribution
(how many brokers), partitioning and the topics (channels) it uses. The
consumer should work and be reconfigurable, if our Kafka configuration
chances

\subsubsection{Allow parallelism for reading from
Kafka}\label{allow-parallelism-for-reading-from-kafka}

Depending on the configuration there can be several options, how
parallel reading from Kafka can be accomplished (consumer groups, many
topics and consumers, etc.). The consumer should be able to work for all
of them or at least the ones we decide for.

\subsubsection{Achieve high performance to not slow down the importing
pipeline}\label{achieve-high-performance-to-not-slow-down-the-importing-pipeline}

As our importers reached very high speeds in producing data items
(\textgreater{} 1 million in few seconds) and many of them can be run in
parallel the consumer itself shell also deliver a good performance.
While of course overall we want to tackle every possible bottleneck, for
the consumer itself it was at first most important that it is not the
bottleneck itself.

\subsubsection{Allow concurency for processing data read from
Kafka}\label{allow-concurency-for-processing-data-read-from-kafka}

A naive strategy of serially reading from the queue, processing the data
and the pushing it to Elasticsearch will most likely be way to slow.
Therefore we must go for a more sophisticated method, that uses
asynchronisity and concurrency to manage a non-blocking higher
performant process.

\subsection{Implementation of the
Requirements}\label{implementation-of-the-requirements}

Because of the named reasons we decided to build one component to
validate the output and push it ---if valid--- to Elasticsearch. There
would have been the possibility to do this in separate steps with
another queue in between as well. This would have meant a lot higher
configuration effort and more resources required so that it seemed good
to join both tasks in one component.

\subsubsection{Bulk insertion}\label{bulk-insertion}

The most important decision was on how to handle the insertion to
Elasticsearch. These requests require an HTTP connection. If we open and
close one HTTP connection per item inserted, the overhead would be
enormous. Therefore we wanted to use Elasticsearch's bulk insertion
feature. The HTTP requests should as well be non-blocking and be
processed in the background to not halt the execution of the other
important tasks, like listening to the queue and validation the data.

\subsubsection{Achieve high performance and allow
validating}\label{achieve-high-performance-and-allow-validating}

In order to fulfill the requirements we went with \emph{Go} as a
programming language. The main reasons for that was the outstanding
performance and the build in concurrency features, which directly
tackled the main requirements. Parallelism, configuration and
scalability was achieved independent from the programming language and
the consumer itself by deploying it the right way. This will be
described in the paragraphs after this about Go-specific advantages.

Although parallelism can be achieved by simply spawning more consumers
(see later) it was still very important to maintain a very fast
performance within one importer. While reading from a queue, validating
and inserting seemed like a typically task for a scripting language, the
performance shortfalls of interpreted languages are too high to use one
for this time-sensitive task. Needless to say we also discovered time
issues with some test-consumers written e.g.~in \emph{ruby}.

For validating the JSON items there is the need for a library that
handles validation of the used \emph{JSON Schema} format. There is at
least one library present for the most relevant programming languages,
so this did not limit us in chosing a programming language.

\subsubsection{Allow concurrency for processing data read from
Kafka}\label{allow-concurrency-for-processing-data-read-from-kafka}

This was another main reason to chose \emph{Go} as a programming
language as concurrency is very well supported by the built-in Go
routines. With this we could achieve a procedure that works like this:

\begin{itemize}
\tightlist
\item
  Continuously listen to the Kafka queue for new item on the topic.
\item
  Directly validating them with a preloaded schema that is already in
  memory and therefore takes nearly no time
\item
  Asynchronously passing the validated JSON to another Go routine, that
  aggregates the items to bulks until

  \begin{itemize}
  \tightlist
  \item
    The bulk limit is reached or
  \item
    A timeout is triggered (from outside the routine, needed if
    importing is very slow, or especially for the last items released by
    the importer)
  \end{itemize}
\item
  the aggregated JSON is again asynchronously sent to another Go
  routine, that handles the HTTP connection to Elasticsearch in the
  background.
\end{itemize}

\subsubsection{Allow parallelism, scaling, and
configuration}\label{allow-parallelism-scaling-and-configuration}

Scaling and configuring the consumer would have been possible in any
language: scaling is a job solved by the deployment on a Kubernetes
cluster and configuring it can be done with environment variables and
the same deployment component setting them accordingly. Therefore the
parallelism required was also not a requirement directly to the consumer
itself.

There are two possibilities to allow parallelized reading from Kafka for
a data source:

\begin{enumerate}
\def\labelenumi{\arabic{enumi}.}
\tightlist
\item
  Using Kafka's Consumer Groups to allow several consumers reading from
  one topic
\item
  Using a topic per importer instance and not per data source (e.g.~when
  several importers run for several days for a data source.)
\end{enumerate}

Both of them have been tested by us and both of them have their
advantages. As Elasticsearch was the limiting component in an end-to-end
importing pipeline by not being able to insert \textgreater{} 5000
records/second with our configuration, we have not been able to do a
proper benchmark on how consumer groups behave performance-wise and if
it can reach the same performance than using a consumer per importing
instance.

We could not achieve to run an even bigger configuration of
Elasticsearch. The consumer therefore supplied a speed that reaches
Elasticsearches limits as soon as two of them are run in parallel, which
can be evaluated as a success and reduces the need of a detailed
benchmark, which options would be more effective.

The consumer is able to join a consumer group, which can be configured
by environment variables and therefore by the deployment component. If a
data source needs several consumers for one importer (e.g.~having lots
of data for an importing interval ---which is typically a day--- that is
produced faster than a consumer can utilize) several of them can be
spawned inside one consumer group for that topic. If one importing
interval can be processed by one importer but there are several of them
run in parallel we can create a topic per importing instance and run a
consumer for each. So the consumer in its current state is capable for
both options.
