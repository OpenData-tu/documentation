\section{Units of Measurement}\label{sec:units}

\vspace{8 pt}
\textbf{Authorship:} Written by Andres Ardila
\vspace{10 pt}


\subsection{Overview}\label{overview}

Since the same physical quantity can be represented in different units
of measurement ---both meters and miles represent lengths, for
example--- storing measurements of physical quantities inherently
introduces the complexity of these different representations. While the
topic is certainly quite interesting, especially when considering how
sensors process and convert signals to digital values for a measurement,
we shall limit the scope of our discussion here 1) to how our system
deals with the different ways in which the same physical quantity can be
represented, and 2) to the facilities provided to users by the system to
convert their measurements from one unit to another.

\subsection{Requirements}\label{requirements}

From early on in the analysis phase, it became apparent that in order to
deal with the heterogeneity of incoming data and to fulfill the
requirement of providing a uniform query interface for the data, a
global schema would have to be introduced and enforced all imported
data. That is, someone who is performing a query would expect the data
to be structurally and semantically consistent, even if it came from
different sources. Say our platform offered ambient temperature
readings; this data may come from different sensors manufactured by
different companies and which provide data in different formats. In
order for the data to be usable for analysis, I, as a data user, would
not expect to have to convert readings from the U.S. to Celsius in order
to be able to compare it with data from Europe, or from degrees Kelvin
if a particular sensor manufacturer has calibrated its devices to report
ambient temperature in this unit.

Having standard units for a given physical quantity means, however, that
data importers must take care to convert all units to the standard for
the platform. In order to reduce friction for our data donators, we also
wish to provide facilities to convert their units to those required by
our system. With this in mind, we formalized our requirements with
respect to units as follows:

\begin{itemize}
\tightlist
\item
  all measurement values stored and reported by our system must be
  accompanied by a unit of measurement
\item
  any measurement values which underwent unit conversion must include
  information about the converter so that it may be reverted, if
  necessary
\item
  provide extensible facilities to users to convert units within their
  importer packages to the required unit defined by the system for that
  physical quantity
\end{itemize}

\subsection{Survey of Previous Work}\label{survey-of-previous-work}

\subsubsection{Standards}\label{standards}

\paragraph{ISO}\label{iso}

ISO 1000 -- \emph{``SI units and recommendations for the use of their
multiples and of certain other units''} was first introduced in 1981 and
revised in 1992, but withdrawn and superseded by ISO(/DIN) 80000 in
2009. The standard is under the ISO/TC 12 technical committee, which is
responsible for

\begin{quote}
``Standardization of units and symbols for quantities and units (and
mathematical symbols) used within the different fields of science and
technology, giving, where necessary, definitions of these quantities and
units. Standard conversion factors between the various units.''
\end{quote}

In addition, now-withdrawn ISO 2955:1983 \emph{``Information processing
-- Representation of SI and other units in systems with limited
character sets''} deals with encoding unit symbols for machine
processing.

Unfortunately, ISO standards are not available free of charge, so their
relevance and usefulness to our project could not be evaluated.

\paragraph{ANSI X3.50}\label{ansi-x3.50}

The 1986 standard \emph{``Representations for U.S. Customary, SI, and
Other Units to Be Used in Systems with Limited Character Sets''} deals
with the symbolic representation of the units, and as a result is not of
particular interest to our objectives, namely:

\begin{quote}
``This standard was not designed for {[}\ldots{}{]} usage by humans as
input to, or output from, data systems. {[}\ldots{}{]} They should never
be printed out for publication or for other forms of public information
transfer.''
\end{quote}

\paragraph{NIST 811}\label{nist-811}

The \emph{Guide for the Use of the International System of Units (SI)}
from 1995 and updated in 2008 provides a comprehensive reference
regarding SI and units in general aimed to assist scientific paper
authors from the National Institute of Standards and Technology (NIST).
The guide proved useful in condensing the wealth of information that is
SI and the variety of units and formats in an straight forward, complete
yet concise document.

\paragraph{Unified Code for Units of Measure
(UCUM)}\label{unified-code-for-units-of-measure-ucum}

Based on ISO 80000, UCUM's ``purpose is to facilitate unambiguous
electronic communication of quantities together with their units.'' Like
ISO 2955 and ANSI X3.50, its focus is on machine-to-machine
communication and encoding of units. Unlike the latter two standards,
which it claims contain numerous name conflicts and are incomplete, UCUM
``provides a single coding system for units that is complete, free of
all ambiguities, and that assigns to each defined unit a concise
semantics.''

The standard is in scope and of (limited) interest to our application

\subsubsection{Applications}\label{applications}

\paragraph{GNU Units}\label{gnu-units}

String-based command-line application for converting units. Available as
Linux and Windows binaries.

\paragraph{jScience}\label{jscience}

This library is, among others, an implementation of the UCUM standard
mentioned prior. The library was part of a Java Specification Request
(JSR) to be made part of the Java Standard Library under JSR-275, which
was rejected in 2010.

The project itself is not in active development and can no longer be
downloaded from their main site, as the source code and binaries were
hosted in the now-defunct java.net platform.

\paragraph{JSR-363 -- Units of Measurement
API}\label{jsr-363-units-of-measurement-api}

Based on jScience (JSR-275), this library provides a rich programming
interface to express quantities and units in Java. The proposal is on
its way to being approved at the time of writing (July, 2017).

\subsubsection{Conclusions of Survey}\label{conclusions-of-survey}

The Units of Measurement API (JSR-363) is promising. However, given the
fact that many such libraries have failed to gain noticeable traction in
the past (at least in Java), and because we want to keep the learning
curve for enthusiast data importers as low as possible, we consider this
API to be too complex for the simple task at hand: to convert units.

\subsection{Implementation}\label{implementation}

\subsubsection{Converters}\label{converters}

First, let us consider the \texttt{Serializable} Java interface. In
serialization, the goal is to transform an in-memory object into a
format which enables its state to be persisted, and conversely to
re-create an object from this persisted state to an in-memory object
again. To accomplish this, the interface requires one method to
serialize the object, and one to deserialize, respectively.
Additionally, because the serialization implementation may have changed
between the time an object was serialized and when it will be
deserialized, the interface imposes a version descriptor (simply a
\texttt{static\ final\ long\ serialVersionUID} value ).

Our task is quite similar to serialization, not in that we seek to
persist an object (which has attributes with values), but rather in that
there is some process which renders a particular instance of a ``thing''
into a different \emph{representation}; additionally, it's also able to
reverse this process. Versioning is naturally of interest as well, since
without some mechanism to revert a conversion, a faulty unit converter
would permanently render converted measurements unusable and
irrecoverable. A converter version would enable us then to revert
incorrect conversions.

At its simplest, a unit converter could be expressed as:

\begin{verbatim}
public absrtact class UnitConverter {

    public abstract double convert(double source);
    public abstract double inverse(double source);
}
\end{verbatim}

And an example Celsius-to-Farenheit converter:

\begin{verbatim}
public class CelsiusToFarenheitUnitConverter extends UnitConverter {
    @Override
    public double convert(double source) {
        return (source * 9d)/5d + 32;
    }

    @Override
    public double inverse(double source) {
        return (source - 32) * 5d/9d;
    }
}
\end{verbatim}

The converters would be made available to the community and open for
contributions. This can be achieved through a public version control
system (such as Git), on which merge requests could be accepted.

\subsection{Discussion}\label{discussion}

\textbf{Authoriship:} Written by Paul, proofread and edited by Andres
\vspace{10 pt}

This implementation has some advantages but also some disadvantages. In
this section we want to take a closer look to both sides.

As we force the user to use our main unit, we ensure that all data in
the database has the same unit for a given measurement type. Of course
we cannot enforce that the user does indeed convert measurements
correctly or at all, but this would be considered a faulty import, which
in the end is the responsibility of the user. Of course an assessment of
the correctness of data would be nice, but this is also hard to achieve
and not within the scope of our project.

Our approach of having a curated list means some management overhead and
possible longer implementation effort for the user if a unit conversion
he or she needs is not yet available. Given the vast number of units in
general and the lack of standardization in the way sensors report their
data, giving a lot of latitude to the user to specify the units and the
necessary conversions seems like the only reasonable way in which to
approach the issue.

As the unit categories should be present after a short testing phase of
a system, and a main unit exists with with, as the curators decide on
one, the user should most of the time be able to register a source, when
he wants to, as he only needs to know the main unit.

A big advantage of our approach is, that we kind of crowd-source the
implementation of converters by this, as it happens during the ETL phase
while importing a source. This gives us a chance to achieve the following:

\begin{itemize}
\tightlist
\item
  Conversions can be reverted, as the converter used is stored with the
  data.
\item
  Localization within our database can easily be done, as all
  measurements of a unit category have the same unit and converters are
  written the moment someone has to convert his source data to our
  preferred unit.
\item
  By crowd-sourcing the implementation of converters they are also open
  sourced for reuse by other users. Having our own converters only in
  the system to convert measurements after they are in the database
  would not guarantee the reusability as importers and our database
  frontend depend on totally different things.
\end{itemize}

\subsection{Future Improvements}\label{future-improvements}

Our implementation in this regard is a proof of concept. The
community-sourcing aspect remains to be implemented, though, as
discussed, a public Git repository would be a feasible low-effort first
alternative.
