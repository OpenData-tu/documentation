\section{Project Management \& Organization}\label{sec:organization}

\textbf{Authorship}

\begin{longtable}[]{@{}llll@{}}
\toprule
Version & Date & Modified by & Summary of changes\tabularnewline
\midrule
\endhead
0.1 & 2017-07-19 & Andres, Oliver & Sections and initial
outline\tabularnewline
1.0 & 2017-07-27 & Oliver & Outline \& first written
draft\tabularnewline
2.0 & 2017-07-28 & Oliver & Rephrased \& extended team
section\tabularnewline
2.1 & 2017-07-29 & Andres & Proofread\tabularnewline
\bottomrule
\end{longtable}

The main challenge of project management is to achieve specific results
within a given scope, time, quality, and budget. At the end of every
successful project stands a unique product, meeting the specified
requirements. This chapter elaborates on the available resources, the
development process and chosen approach. The resources that this chapter
will focus on are the heterogeneous team and limited budget.

\subsection{The Team}\label{the-team}

The project team consisted of seven masters students majoring in
Computer Science, Computer Engineering and Information Systems
Management. Moreover, the team members are native of five different
countries with different cultural backgrounds and native languages. Due
to varying English skills and professional experience, some discussion
resulted to be time-consuming. In addition, most team members work besides
being students, hence schedules had to be aligned accordingly.

The team was organized into expert groups for the various building
blocks with one person in a coordinative role as the project manager.
Thus, a hierarchy was formed within the group that was hard to enforce
since, in the end, the team was formed of peers. In the beginning, two
teams had been formed: one that was responsible for the Importing
Framework and data sources, and one for the Database and Infrastructure.
As the project progressed, however, the roles of several members shifted. In
the end, one member focused on the modeling and administration of data
sources, another member focused on the infrastructure provisioning and
deployment, the team responsible for the importing framework consisted
of three persons and lastly, one team two person team worked on the
backend.

\subsection{Constraints}\label{constraints}

Like every other project, this one was subject to several constraints.
The very nature of a project makes it a temporary endeavor. Thus, the
product had to be finished within eleven weeks starting April 27 and
ending July 13. Additionally, towards the end of the first half June 1,
an interim meeting was held. During these eleven weeks, three feedback
sessions were held, one of them before the interim meeting and two after.
After July 13, two more weeks were available to write documentation.

Moreover, monetary restrictions added an extra level of difficulty.
While developing software consisting of several disparate building
blocks, it is crucial to deploy and test the components individually and
collectively. Due to a lack of an own infrastructure, the only feasible
way to conduct testing ---in functionality and performance--- was to use
a commercial cloud provider. Owing to no available budget, the decision
for the US\$100 student grant offered by Amazon Webservices (AWS) was made.
Nevertheless, this added even more effort to receive access to the
student grant and, furthermore, to also be able to use all of the
services provided. Some members did not receive any free credit from AWS.

Moreover, the budget constraints forced the team to start and terminate
the whole infrastructure every time a test was conducted. In other
words, a significant amount of time was necessary to create the test
environment each time, which put additional strain with regards to
the already strict time limit. Hence, a significant effort was put into
automating processes in order to provision the infrastructure and
deployment of components, and then shutting them down to prevent the
budget from depleting.

Having dedicated infrastructure from the beginning would have allowed to
focus more on the actual challenges of the project itself.

\subsection{Development Process \&
Approach}\label{development-process-approach}

To best commence the given challenges and requirements, we decided for
an agile development process. To be more specific, we used a
stripped-down Scrum approach. In this approach, the distinction between
different roles was not strict so that every team member held
simultaneously the role of a product owner and developer for his
specific building block of the global architecture. The same is
applicable for the role of Scrum Master, which was shared among the
whole team, but still was mainly the responsibility of the Project
Manager. Specifically, this means that every team member was encouraged
to ensure an agile process and administrate issues. Furthermore, the
workflow was adapted to our specific scheduling needs so that we convened a
sprint meeting weekly with an additional weekly interim meeting and
omitted the daily scrum.

To keep track of every issue and the global progress, we used a Trello board.
Trello follows the Kanban model and arranges issues into lists. We created
four lists to mirror the lifecycle of issues. The first list --product
backlog-- is where requirements, issues, and general tasks that the product
must satisfy were created. This list was populated continuously as the project
progressed. The next lists included: tasks due until the next sprint meeting,
tasks that were currently being worked on, and finally, finished tasks
were archived.

Slack was the chosen medium of communication, because it allows to
distinguish between independent topics by using channels. Additionally,
a shared calendar was used to remember deadlines and a shared file
system to store our meeting minutes. More importantly, we used
GitHub to store the code written. We formed an \emph{organization} and created
a separate repository for every component.

In order to integrate the tools we used as well as to automate
workflows, a service called Zapier was used. We used it to receive
notifications about changes on GitHub and Google Drive, and the progress
made on Trello. Though, after the trial ended the functionality was
limited so that only notifications about GitHub updates were sent.
