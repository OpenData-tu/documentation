\section{Architecutre}\label{architecutre}

\textbf{Authoriship}

\begin{longtable}[]{@{}llll@{}}
\toprule
Version & Date & Modified by & Summary of changes\tabularnewline
\midrule
\endhead
0.1 & 2017-07-19 & Oliver, Amer & Working draft\tabularnewline
0.2 & 2017-07-28 & Oliver & Tidy up \ldots{}\tabularnewline
\bottomrule
\end{longtable}

\begin{itemize}
\tightlist
\item
  Components Overview/Description (if applicable
  \textbf{\emph{Motivation}})
\item
  Requirements (specific to this component)
\item
  Survey of Existing Solutions (available implementations)
\item
  Evaluation Criteria \& Decision-making Process
\item
  Implementation Details
\item
  Evolution of Component during development (Reasons for the Changes)
\item
  Critical Analysis/Limitations
\item
  Future Development/Enhancements
\end{itemize}

\subsection{Requirements of the whole
platform}\label{requirements-of-the-whole-platform}

\begin{itemize}
\tightlist
\item
  scalability
\item
  define scalability/ our interpretation of this requirement
  specifically for our platform
\item
  extensibility
\item
  define it!
\item
  objective was to create a whole pipeline
\item
  Collection data from various sources
\item
  Process/transform the data
\item
  Store the data persistently
\item
  Provide the data through a ``single'' interface
\item
  no usage of cloud/provider specific solutions
\end{itemize}

\subsection{Design decisions}\label{design-decisions}

To tackle all of the mentioned requirements, we decided upon the
following architecture {[}image of architecture{]}. - We traded
consistency for availability. (Harvest over yield \{\{ref\}\}) -
``Decentralized system'': To be able to scale the platform horizontally,
it has to be distributed. Thus, smaller pieces of the system have to
work on different/separate machines. We omitted bottlenecks by picking
components that are distributed by design.

\subsubsection{Importers}\label{importers}

\begin{itemize}
\tightlist
\item
  Every source is different! Heterogeneous interfaces, different data
  formats, different protocols
\item
  Sources have specific limitations, e.g.~number of requests per
  specific time slot
\item
  Size and frequency of data points is heterogeneous
\item
  Availability of sources is different --\textgreater{} Every importer
  runs as a independent service. All have specific lifecycles (run-time,
  frequency of execution). Possibility to schedule them independently.
  --\textgreater{} Upon finishing the importing task, the running
  container is terminated to allow for maximal resource utilization.
  --\textgreater{} Every importer manages its state independently from
  the system. If the importer failed completing the importing task, it
  continues from the last checkpoint. --\textgreater{} Every importer
  has a unique name when scheduled to avoid repetition of the same
  importing job.
\end{itemize}

\subsubsection{Messaging System}\label{messaging-system}

\begin{itemize}
\tightlist
\item
  Messaging enabled the component to decouple the services from each
  other. We used queue systems to transport and buffer messages to
  decouple the components even further.
\item
  Used a distributed messaging systems that is capable of handling
  millions of concurrent requests and is fault-tolerant.
\item
  Autonomous parts can be deployed independently, such that the platform
  keeps running without interruption in contrast to deploying a
  monolithic application.
\item
  Choice of open source solutions to ease portability and make it
  cross-platform
\end{itemize}

\subsection{Evolution of the
architecture}\label{evolution-of-the-architecture}

\begin{itemize}
\item
  Filebeat got obsolete
\item
  Scheduling handeled by Kubernetes
\item
  Validation and insert into one component
\item
\end{itemize}

\subsection{Limitations}\label{limitations}

\begin{itemize}
\tightlist
\item
  ???
\end{itemize}

\subsection{Future}\label{future}

\begin{itemize}
\item
  Connection from public API to relational system
\item
\end{itemize}
