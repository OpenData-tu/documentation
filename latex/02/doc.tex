\section{Survey of Current Landscape}\label{survey-of-current-landscape}

\vspace{8 pt}
\textbf{Authorship:} Written by Andres Ardila
\vspace{10 pt}

\subsection{OpenSensors.io}\label{opensensors.io}

https://www.opensensors.io

\subsubsection{Features}\label{features}

\begin{quote}
``Securely manage your private IoT network {[}\ldots{}{]} whether you
are a startup or an enterprise''
\end{quote}

\begin{itemize}
\tightlist
\item
  IoT device management
\item
  Real-time and historical APIs
\item
  Analytics
\item
  Ability to run infrastructure in user's own dedicated cloud network
\item
  Integration with user's backend systems
\item
  Secure data sent to the cloud using TLS
\item
  Set policies on who can sees the data by users or groups of engineers
\item
  Triggers when data hits certain thresholds and alert engineers when
  device is down
\end{itemize}

\paragraph{Hardware agnostic}\label{hardware-agnostic}

Provides support for open protocols, use existing SDKs to quickly get up
and running using MQTT or HTTPS protocols. Their partnership program
includes hardware providers and manufacturing firms.

\subsubsection{Segments}\label{segments}

\begin{itemize}
\tightlist
\item
  \textbf{Workspace planning}: ``Use sensors and our data platform to
  understand if you are using it efficiently and forecast your future
  needs.''
\item
  \textbf{Environment sensing}: Open Data communities to contribute and
  use data from environmental sensors around you. Air quality and traffic data
  available ``free''.
\end{itemize}

\subsubsection{Customers}\label{customers}

\textbf{Partners} are hardware manufacturers and design firms (OEMs)
looking to deliver client projects.

\subsubsection{Analysis}\label{analysis}

Not really ``open''. There is no sign-up with which one can get access to
the allegedly ``open'' data. As a result, one can conclude that their
focus is the OEM sector as opposed to ``data scientists looking to
create interesting mash-ups of data.''


\subsection{Plenario}\label{plenario}

http://plenar.io/

\subsubsection{Features}\label{plenario-features}

\begin{quotation}
``Plenario is a centralized hub for open datasets from around the world,
ready to search and download.''
\end{quotation}

\begin{itemize}
\tightlist
\item
  Open platform on which to add open datasets in CSV or ESRI shapefiles
\item
  Only requires data to have time and location information in the datapoints
\item
  Modern open source data visualization for timeseries geolocated data (heatmap, line charts)
\end{itemize}

\subsubsection{Customers}
\textbf{Cities} who already generate and publish open data and wish to provide a rich
interface to users to visualize data from various datasets.

\subsection{Analysis}
Plenario is precisely the next step forward from static data platforms like CKAN.
By requiring that data must contain time and location information, an elegant,
unified, and intuitive visualization is immediately made possible. Unfortunately
it offers no rich querying model, other than filtering by date ranges, but
the concept of `layering' is already present, by allowing the user to visualize
data from more than one data source for analysis on the user interface.
This is definitely a step in the right direction and most importantly, it's open source.

\subsection{PubNub}\label{pubnub}

https://www.pubnub.com/

\subsubsection{Features}\label{features-1}

\begin{quotation}
``The Programmable Data Stream Network
Low-latency messaging. Massive scale. Ready for the real world.''
\end{quotation}

As an IoT end-to-end publish-subscribe messaging and streaming platform,
we are concerned only with their `Storage \& Playback' product.

\begin{itemize}
\tightlist
\item
  Scalability: 15 globally replicated points of presence transacting over 1.5 trillion
messages for 300 million unique devices per month.
\item
  Unlimited storage
\item
  Configurable retention period
\item
  Message-level granularity
\end{itemize}

%\subsubsection{Segments}\label{segments-1}

%\subsubsection{Customers}\label{customers-1}

\subsubsection{Analysis}\label{analysis-1}

PubNub provides robust messaging infrastructure, which coupled with their storage
solution could be seen as providing (at least at a very low level) the piping
for the import part of our system. The programmer is given an lambda-like
programming model in which each event can be processed. For live streaming
sensor data this could be a viable solution, as the lambda could be seen as
performing the data integration task of schema conversion between the sensor
data model and our internal schema to be then be stored. For historical
importers, however, this would not be an ideal model. Unfortunately, since
it is not open source, there are no details available as to the storage engine
or internal architecture, which could be of value for our development as inspiration.

\subsection{Talend}\label{talend}

https://www.talend.com/resource/sensor-data/

\subsubsection{Features}\label{features-2}

Talend is primarily a data integration solution. Their offering for sensor data
revolves around providing a ``an easy-to-use graphical environment, [where] developers
can visually map sensor data sources and targets, and quickly perform complex
transformations and analyses to produce actionable intelligence and drive
performance improvement.''

Talend also provides pre-built connectors for Hadoop database solutions like
Cassandra, CouchDB, Couchbase, HBase, MongoDB, Neo4J and Riak .

\subsubsection{Analysis}\label{analysis-2}

Talend's offering concerns the data integration aspect of our system, so
it is not comparable when considering the system as a whole. Talend is
positioned in the Spark + Hadoop `Big Data` space and as such targets the
leveraging of data to generate analytics. While their data connectors are
an attractive feature in general, our users most likely won't be providing
data through a database connection, but rather through files, so it's
unfortunately not a particularly good fit. Their data integration IDE would
certainly be a nice feature for those users who are not necessarily programmers
but rather simply enthusiasts; a graphical interface to perform a simple
schema transformation would be sufficient for them. Further, storage is provided
as a subscription product as opposed to being an open source solution which
we could peer into for inspiration.
